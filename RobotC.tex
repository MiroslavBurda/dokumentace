
\documentclass[12pt]{article}
\usepackage[czech]{babel} % nastavuje české popisky např. u obsahu, referencí, tabulek, obázků 
\usepackage[utf8]{inputenc} % použito UTF8 kvůli češtině (zvládá prakticky všechny jazyky na světě)


%\usepackage{indentfirst} % odsazuje první odstavec v kapitole

\usepackage{color} % balíček pro obarvování textů
% \color{blue}
\usepackage{xcolor}  % zapne možnost používání barev, mj. pro \definecolor
\definecolor{mygreen}{RGB}{0,150,0} % nastavení barev odkazů 
\definecolor{myblue}{RGB}{0,0,200} 
\definecolor{commentgreen}{RGB}{0,100,0} % nastavení barev pro příklady z C++
\definecolor{deepblue}{rgb}{0,0,0.7}
\definecolor{deepred}{rgb}{0.6,0,0}
\definecolor{deepgreen}{rgb}{0,0.5,0}

\usepackage{hyperref} % balíček pro hypertextové odkazy
% \url{www.odkaz.cz}
% \href{http://www.odkaz.cz}{Text který bude jako odkaz}
%\hyperlink{label}{proklikávací_text} - odkaz na text 
% \hypertarget{label}{cíl_odkazu} - cíl odkazu  


\hypersetup{colorlinks=true, linkcolor=myblue, urlcolor=mygreen, citecolor=blue, anchorcolor = magenta,
	linktocpage = true, frenchlinks } % nastavení barvy odkazů 
% bookmarksopen=true, bookmarksnumbered=true, bookmarksopenlevel=1 - nastavuje rozbalování levého menu       

\usepackage{graphicx} % pro vkládání obrázků a příkaz "\includegraphics"
%% samotné vložení obrázku
%\begin{figure}
%	\includegraphics[width=\textwidth]{../img/when_use_latex.png}
%	\caption{Kdy se vyplatí použít \LaTeX} % popis, který se zobrazí pod obrázkem
%	\label{moje_navesti} %identifikuje objekt, který lze pak referencovat
%\end{figure}
%% ---------------------------

% Délky a mezery pro celý dokument
\setlength{\parskip}{0.5ex plus 0.5ex minus 0.2ex} %Nastavuje velikost vertikální mezery mezi odstavci:
% první číslo je mezera mezi odstavci, druhé nevím a třetí je mezera před začátkem nové kapitoly a podkapitoly
\parindent=0pt % odstavce nebudou odsazeny zarážkou (velikost odsazení prvního řádku odstavce)
\usepackage{enumitem} % pro příkaz \setlist
\setlist{itemsep = -1pt, topsep=3pt} % nastavuje mezery mezi a před výčtovým prosředím (enumerate, itemize ...)

%%%%%%%%%%%%%%%%%%%%%%%%%%%%%%%%%%%%%%%%%%%%%%%%%%%%%%%%%%%%%%%%%%%%%%%%%


\begin{document}
	\begin{center}
		\Huge \textbf{RobotC} 
	\end{center}

\section{RobotC}

\subsection{Začátek}
\begin{enumerate}
	\item \textbf{stáhnout a nainstalovat program} ze stránek \href{http://www.robotc.net/download/lego/}{robotc.net}
	\item \textbf{vybrat platformu EV3:} \\
	 \textless Robot\textgreater \textless Platform type\textgreater \textless Lego Mindstorms\textgreater \textless Lego Mindstorms EV3\textgreater 
	\item \textbf{nahrát do kostky Kernel}: \textless Robot\textgreater \textless Download linux Kernel\textgreater  \\
	(musí být přes kabel připojená a zapnutá kostka!)
	\item  \textbf{nahrát do kostky firmware} ve verzi dodané staženým ROBOTC: \\
	\textless Robot\textgreater \textless Download Firmware .... \textgreater 
	
	\item \textbf{nastavit motory a senzory} tak, jak jsou připojené ke kostce:\\
	\textless Robot\textgreater \textless Motors and sensors setup\textgreater  \\ a na záložkách \textit{Motors} a \textit{Sensors} nastavit potřebné hodnoty
	
\end{enumerate}
  
\subsection{Programování}
  
\begin{itemize}
	\item programovat se může textově nebo graficky, pro úplné začátečníky je jednodušší graficky (samostatná ikona)
	\item pro nahrávání programu musí být kostka zapnutá
	\item začněte s volbou \texttt{New file}
	\item vlevo v menu je dostupná většina možných příkazů
	\item nastavení tabulátoru na 4: 
	\texttt{View > Preferences > Detailed preferences > Editor > Text > Tab size}
	\item \texttt{Natural language} znamená už předchystané funkce -- dají se rozklíčovat, když se na ně klikne pravým tlačítkem a vybere se \texttt{Go to definition /declaration} (pro ostatní funkce funguje taky) 
	\item \texttt{<F5>} mě nefunguje, každý nový program/úprava programu se musí napřed zkompilovat (\texttt{Compile program} nahoře na liště), potom nahrát do robota (\texttt{Download to robot} hned vedle) a nakonec spustit (buď tlačítko \texttt{Start} v okně \textit{Debugger} (okno \texttt{Debugger} se otevře samo po nahrání programu do robota)  nebo na kostce v adresáři \texttt{rc})
	\item velmi praktické volby \texttt{Robot > Debugger Windows} se zpřístupní až po nahrání programu do robota
	
	\item umí vícevlákonové programování, vlákna nazývá tasks (max. 20 vláken), vlákno se musí spustit (\texttt{startTask}) a na konci programu zastavit (\texttt{stopTask}), více v helpu \texttt{F1}, sekce \texttt{ROBOTC > Task Control}, dále v 
	\href{http://georgegillard.com/documents/1-the-beginners-guide-to-robotc}{The Beginners Guide to ROBOTC}, díl 2 kapitola 4 (jednoduché) a 
	\href{https://www.vexforum.com/t/discussion-on-using-tasks-in-robotc/33025}{jpearman} (pokročilé a podrobné)
	
	
\end{itemize}
  
  
\subsection{Ostatní}


\begin{itemize}
	\item \href{http://help.robotc.net/WebHelpMindstorms/index.htm}{List of Commands for EV3}
	\item VADÍ diakritika v názvech souborů (a asi i v názvech proměnných) $\rightarrow$ program nejde nahrát do kostky
	\item  napětí na baterii lze zjistit z menu \texttt{Robot > Debugger Windows > System Parameters} (objeví se v dolním okně)
	\item RobotC lze s EV3 kostkou propojit přes bluetooth, postup: 
	\begin{enumerate}
		\item zapnout kostku a povolit v ní Bluetooth Nastavení (ikona vpravo nahoře) v ní Bluetooth, položky \texttt{Bluetooth} i \texttt{Visibility} musí být zaškrtnuté 
		\item připojit kostku k PC a spárovat 
		\item  v programu RobotC: \texttt{Robot > Lego Brick > Communication Link Setup} po této volbě se program pokusí kostku najít a když se mu to povede (kostka se objeví v okně \texttt{Select Brick to Use ...}),
		 tak je komunikace zprovozněná a chová se úplně stejně jako přes kabel (jen to déle trvá -- sekundy)   
	\end{enumerate}
	
\end{itemize}


 
 \end{document}